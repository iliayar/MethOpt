% Created 2021-04-20 Tue 19:43
% Intended LaTeX compiler: pdflatex

\documentclass[english]{article}
\usepackage[T1, T2A]{fontenc}
\usepackage[lutf8]{luainputenc}
\usepackage[english, russian]{babel}
\usepackage{minted}
\usepackage{graphicx}
\usepackage{longtable}
\usepackage{hyperref}
\usepackage{xcolor}
\usepackage{natbib}
\usepackage{amssymb}
\usepackage{stmaryrd}
\usepackage{amsmath}
\usepackage{caption}
\usepackage{mathtools}
\usepackage{amsthm}
\usepackage{tikz}
\usepackage{grffile}
\usepackage{extarrows}
\usepackage{wrapfig}
\usepackage{algorithm}
\usepackage{algorithmic}
\usepackage{lipsum}
\usepackage{rotating}
\usepackage{placeins}
\usepackage[normalem]{ulem}
\usepackage{amsmath}
\usepackage{textcomp}
\usepackage{capt-of}


\usepackage{geometry}
\geometry{a4paper,left=2.5cm,top=2cm,right=2.5cm,bottom=2cm,marginparsep=7pt, marginparwidth=.6in}
 \usepackage{hyperref}
 \hypersetup{
     colorlinks=true,
     linkcolor=blue,
     filecolor=orange,
     citecolor=black,      
     urlcolor=blue,
     }

\date{}
\title{}
\hypersetup{
 pdfauthor={},
 pdftitle={},
 pdfkeywords={},
 pdfsubject={},
 pdfcreator={Emacs 28.0.50 (Org mode 9.4.4)}, 
 pdflang={English}}
\begin{document}

\begin{titlepage}
  \begin{center}
    \large\textbf{Федеральное государственное автономное образовательное учреждение высшего образования ``Национальный исследовательский университет ИТМО``} \\
    \vspace{0.5cm}
    Факультет информационных технологий и программирования \\
    \vspace{0.5cm}
    Направление ``Прикладная математика и информатика`` \\
    \vspace{3cm}
    Отчет к лабораторной работе №4 \\
    \vspace{0.5cm}
    \textbf{Изучение алгоритмов метода Ньютона и его модификаций}
  \end{center}
  \vfill
  \begin{flushright}
    \large
    Выполнили студенты группы М3237 \\
    \vspace{0.5cm}
    Ярошевский Илья \\
    Аникина Вероника \\
    Крюков Александр
  \end{flushright}
  \vspace{3cm}
  \begin{center}
    Санкт-Петербург 2021
  \end{center}
\end{titlepage}
\section{Цели работы}
Реализовать и ислледовать:
\begin{itemize}
\item методы Ньютона:
  \begin{itemize}
  \item классический
  \item с одномерным поиском
  \item с направлением спуска
  \end{itemize}
\item метод Бройдена-Флетчера-Шено и метод Пауэлла
\item метод Марквардта двумя вариантами
\end{itemize}
\section{Ход работы}
\section{Выводы}
\end{document}
