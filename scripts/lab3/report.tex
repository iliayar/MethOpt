% Created 2021-04-20 Tue 19:43
% Intended LaTeX compiler: pdflatex

\documentclass[english]{article}
\usepackage[T1, T2A]{fontenc}
\usepackage[lutf8]{luainputenc}
\usepackage[english, russian]{babel}
\usepackage{minted}
\usepackage{graphicx}
\usepackage{longtable}
\usepackage{hyperref}
\usepackage{xcolor}
\usepackage{natbib}
\usepackage{amssymb}
\usepackage{stmaryrd}
\usepackage{amsmath}
\usepackage{caption}
\usepackage{mathtools}
\usepackage{amsthm}
\usepackage{tikz}
\usepackage{grffile}
\usepackage{extarrows}
\usepackage{wrapfig}
\usepackage{algorithm}
\usepackage{algorithmic}
\usepackage{lipsum}
\usepackage{rotating}
\usepackage{placeins}
\usepackage[normalem]{ulem}
\usepackage{amsmath}
\usepackage{textcomp}
\usepackage{capt-of}


\usepackage{geometry}
\geometry{a4paper,left=2.5cm,top=2cm,right=2.5cm,bottom=2cm,marginparsep=7pt, marginparwidth=.6in}
 \usepackage{hyperref}
 \hypersetup{
     colorlinks=true,
     linkcolor=blue,
     filecolor=orange,
     citecolor=black,      
     urlcolor=blue,
     }

\date{}
\title{}
\hypersetup{
 pdfauthor={},
 pdftitle={},
 pdfkeywords={},
 pdfsubject={},
 pdfcreator={Emacs 28.0.50 (Org mode 9.4.4)}, 
 pdflang={English}}
\begin{document}

\begin{titlepage}
    \begin{center}
        \large\textbf{Федеральное государственное автономное образовательное учреждение высшего образования ``Национальный исследовательский университет ИТМО``} \\
        \vspace{0.5cm}
        Факультет информационных технологий и программирования \\
        \vspace{0.5cm}
        Направление ``Прикладная математика и информатика`` \\
        \vspace{3cm}
        Отчет к лабораторной работе №3 \\
        \vspace{0.5cm}
        \textbf{Методы решения систем линейных уравнений}
    \end{center}
    \vfill
    \begin{flushright}
        \large
        Выполнили студенты группы М3237 \\
        \vspace{0.5cm}
        Ярошевский Илья \\
        Аникина Вероника \\
    Крюков Александр
    \end{flushright}
    \vspace{3cm}
    \begin{center}
        Санкт-Петербург 2021
    \end{center}
\end{titlepage}

\section{Цели работы}

\begin{enumerate}
    \item Реализовать прямой метод решения СЛАУ на основе \(LU\)-разложения
    \item Провести исследование метода на матрицах, число обусловленности которых регулируется за счёт изменения диагонального  преобладания 
    \item Провести исследование метода на матрицых Гильберта различной размерности
    \item Реализовать метод Гаусса с выбором ведущего элемента для плотных матриц
\end{enumerate}
\section{Ход работы}
\subsection{Прямой метод}
\subsubsection{Тестирование на матрицах с диагональным преобладание}
\begin{center}
  \begin{longtable}{l|l|l|l}
    \(n\) & \(k\) & \(\Vert x^* - x_k \Vert\) & \(\frac{\Vert x^* - x_k \Vert}{\Vert x^* \Vert}\) \\
    \hline
10 & 0 & \(5.46103\cdot 10^{-14} \)& \(2.7832\cdot 10^{-15}\) \\
10 & 1 & \(1.43296\cdot 10^{-13} \)& \(7.30302\cdot 10^{-15}\) \\
10 & 2 & \(3.46274\cdot 10^{-12} \)& \(1.76478\cdot 10^{-13}\) \\
10 & 3 & \(1.29306\cdot 10^{-11} \)& \(6.59007\cdot 10^{-13}\) \\
10 & 4 & \(3.44677\cdot 10^{-10} \)& \(1.75664\cdot 10^{-11}\) \\
10 & 5 & \(1.29248\cdot 10^{-9} \)& \(6.5871\cdot 10^{-11}\) \\
10 & 6 & \(3.30301\cdot 10^{-8} \)& \(1.68337\cdot 10^{-9}\) \\
10 & 7 & \(5.74436\cdot 10^{-7} \)& \(2.92759\cdot 10^{-8}\) \\
10 & 8 & \(8.32931\cdot 10^{-6} \)& \(4.24501\cdot 10^{-7}\) \\
10 & 9 & \(1.43609\cdot 10^{-6} \)& \(7.31898\cdot 10^{-8}\) \\
\hline
50 & 0 & \(1.14086\cdot 10^{-12} \)& \(5.50651\cdot 10^{-15}\) \\
50 & 1 & \(4.26714\cdot 10^{-13} \)& \(2.0596\cdot 10^{-15}\) \\
50 & 2 & \(2.93195\cdot 10^{-10} \)& \(1.41515\cdot 10^{-12}\) \\
50 & 3 & \(2.0648\cdot 10^{-9}  \)& \(9.96604\cdot 10^{-12}\) \\
50 & 4 & \(6.88083\cdot 10^{-9} \)& \(3.32113\cdot 10^{-11}\) \\
50 & 5 & \(1.4312\cdot 10^{-7}  \)& \(6.90788\cdot 10^{-10}\) \\
50 & 6 & \(1.83487\cdot 10^{-8} \)& \(8.85625\cdot 10^{-11}\) \\
50 & 7 & \(2.38532\cdot 10^{-5} \)& \(1.15131\cdot 10^{-7}\) \\
50 & 8 & \(0.000193578 \)& \(9.34332\cdot 10^{-7}\) \\
50 & 9 & \(0.00413754  \)& \(1.99704\cdot 10^{-5}\) \\
\hline
100 & 0 & \(9.65618\cdot 10^{-12} \)& \(1.66005\cdot 10^{-14}\) \\
100 & 1 & \(1.06903\cdot 10^{-10} \)& \(1.83783\cdot 10^{-13}\) \\
100 & 2 & \(6.80943\cdot 10^{-10} \)& \(1.17065\cdot 10^{-12}\) \\
100 & 3 & \(9.96402\cdot 10^{-9} \)& \(1.71298\cdot 10^{-11}\) \\
100 & 4 & \(4.38007\cdot 10^{-8} \)& \(7.53005\cdot 10^{-11}\) \\
100 & 5 & \(1.54526\cdot 10^{-6} \)& \(2.65656\cdot 10^{-9}\) \\
100 & 6 & \(1.16246\cdot 10^{-6} \)& \(1.99846\cdot 10^{-9}\) \\
100 & 7 & \(6.24677\cdot 10^{-5} \)& \(1.07392\cdot 10^{-7}\) \\
100 & 8 & \(0.0003323   \)& \(5.71277\cdot 10^{-7}\) \\
100 & 9 & \(0.00224864  \)& \(3.86579\cdot 10^{-6}\) \\
\hline
500 & 0 & \(1.59548\cdot 10^{-11} \)& \(2.46801\cdot 10^{-15}\) \\
500 & 1 & \(3.26512\cdot 10^{-9} \)& \(5.05073\cdot 10^{-13}\) \\
500 & 2 & \(1.2449\cdot 10^{-7}  \)& \(1.9257\cdot 10^{-11}\) \\
500 & 3 & \(5.4828\cdot 10^{-8}  \)& \(8.4812\cdot 10^{-12}\) \\
500 & 4 & \(1.14655\cdot 10^{-5} \)& \(1.77358\cdot 10^{-9}\) \\
500 & 5 & \(6.59951\cdot 10^{-5} \)& \(1.02086\cdot 10^{-8}\) \\
500 & 6 & \(0.000661574 \)& \(1.02337\cdot 10^{-7}\) \\
500 & 7 & \(0.0112052   \)& \(1.73331\cdot 10^{-6}\) \\
500 & 8 & \(0.0891671   \)& \(1.3793\cdot 10^{-5}\) \\
500 & 9 & \(0.635155    \)& \(9.82504\cdot 10^{-5}\) \\
\hline
1000 & 0 & \(2.48326\cdot 10^{-8} \)& \(1.35912\cdot 10^{-12}\) \\
1000 & 1 & \(1.55912\cdot 10^{-8} \)& \(8.53328\cdot 10^{-13}\) \\
1000 & 2 & \(3.0785\cdot 10^{-7}  \)& \(1.6849\cdot 10^{-11}\) \\
1000 & 3 & \(1.72248\cdot 10^{-5} \)& \(9.42737\cdot 10^{-10}\) \\
1000 & 4 & \(2.15435\cdot 10^{-5} \)& \(1.1791\cdot 10^{-9}\) \\
1000 & 5 & \(0.00109214  \)& \(5.97744\cdot 10^{-8}\) \\
1000 & 6 & \(0.0157364   \)& \(8.6127\cdot 10^{-7}\) \\
1000 & 7 & \(0.0697018   \)& \(3.81486\cdot 10^{-6}\) \\
1000 & 8 & \(0.843813    \)& \(4.61829\cdot 10^{-5}\) \\
1000 & 9 & \(1.70491     \)& \(9.33118\cdot 10^{-5}\)
  \end{longtable}
\end{center}
\subsubsection{Тестирование на матрицах Гильберта}
\begin{center}
  \begin{longtable}{l|l|l}
    \(n\) & \(\Vert x^* - x_k \Vert\) & \(\frac{\Vert x^* - x_k \Vert}{\Vert x^* \Vert}\) \\
    \hline
    10 & \(8.25036\cdot 10^{-9}\) & \(4.20477\cdot 10^{-10}\) \\
    50 & \(7.52844\cdot 10^{-7}\) & \(3.63371\cdot 10^{-9}\) \\
    100 & \(3.95193\cdot 10^{-6}\) & \(6.79401\cdot 10^{-9}\) \\
    500 & \(0.000432803\) & \(6.69491\cdot 10^{-8}\) \\
    1000 & \(0.00435918 \) & \(2.38583\cdot 10^{-7}\) 
  \end{longtable}
\end{center}
\subsection{Метод Гаусса с выбором ведущего элемента}
\subsubsection{Тестирование на матрицах с диагональным преобладание}
\begin{center}
  \begin{longtable}{l|l|l|l}
    \(n\) & \(k\) & \(\Vert x^* - x_k \Vert\) & \(\frac{\Vert x^* - x_k \Vert}{\Vert x^* \Vert}\) \\
    \hline
    10 & 0 & \(6.32562\cdot 10^{-14}\) & \(3.22384\cdot 10^{-15}\) \\
    10 & 1 & \(1.60765\cdot 10^{-13}\) & \(8.19336\cdot 10^{-15}\) \\
    10 & 2 & \(2.30718\cdot 10^{-12}\) & \(1.17585\cdot 10^{-13}\) \\
    10 & 3 & \(1.2928\cdot 10^{-11}\) & \(6.58874\cdot 10^{-13}\) \\
    10 & 4 & \(7.32431\cdot 10^{-10}\) & \(3.73281\cdot 10^{-11}\) \\
    10 & 5 & \(4.16467\cdot 10^{-9}\) & \(2.12251\cdot 10^{-10}\) \\
    10 & 6 & \(7.18045\cdot 10^{-9}\) & \(3.6595\cdot 10^{-10}\) \\
    10 & 7 & \(1.43609\cdot 10^{-7}\) & \(7.31899\cdot 10^{-9}\) \\
    10 & 8 & \(5.45714\cdot 10^{-6}\) & \(2.78121\cdot 10^{-7}\) \\
    10 & 9 & \(1.00526\cdot 10^{-5}\) & \(5.12329\cdot 10^{-7}\) \\
    \hline
    50 & 0 & \(3.39627\cdot 10^{-12}\) & \(1.63926\cdot 10^{-14}\) \\
    50 & 1 & \(7.07182\cdot 10^{-11}\) & \(3.41331\cdot 10^{-13}\) \\
    50 & 2 & \(6.73371\cdot 10^{-10}\) & \(3.25012\cdot 10^{-12}\) \\
    50 & 3 & \(1.86195\cdot 10^{-8}\) & \(8.98695\cdot 10^{-11}\) \\
    50 & 4 & \(5.47724\cdot 10^{-8}\) & \(2.64366\cdot 10^{-10}\) \\
    50 & 5 & \(9.17442\cdot 10^{-9}\) & \(4.42816\cdot 10^{-11}\) \\
    50 & 6 & \(4.27524\cdot 10^{-6}\) & \(2.0635\cdot 10^{-8}\) \\
    50 & 7 & \(3.20184\cdot 10^{-5}\) & \(1.54541\cdot 10^{-7}\) \\
    50 & 8 & \(0.0010422\) & \(5.03031\cdot 10^{-6}\) \\
    50 & 9 & \(0.00196329\) & \(9.4761\cdot 10^{-6}\) \\
    \hline
    100 & 0 & \(9.19552\cdot 10^{-12}\) & \(1.58086\cdot 10^{-14}\) \\
    100 & 1 & \(1.67913\cdot 10^{-10}\) & \(2.8867\cdot 10^{-13}\) \\
    100 & 2 & \(5.97625\cdot 10^{-11}\) & \(1.02741\cdot 10^{-13}\) \\
    100 & 3 & \(3.8314\cdot 10^{-9}\) & \(6.5868\cdot 10^{-12}\) \\
    100 & 4 & \(5.50153\cdot 10^{-8}\) & \(9.45803\cdot 10^{-11}\) \\
    100 & 5 & \(2.62084\cdot 10^{-6}\) & \(4.50565\cdot 10^{-9}\) \\
    100 & 6 & \(9.08835\cdot 10^{-6}\) & \(1.56243\cdot 10^{-8}\) \\
    100 & 7 & \(0.000175778\) & \(3.02191\cdot 10^{-7}\) \\
    100 & 8 & \(0.000469095\) & \(8.06451\cdot 10^{-7}\) \\
    100 & 9 & \(0.00579465\) & \(9.96195\cdot 10^{-6}\) \\
    \hline
    500 & 0 & \(4.74104\cdot 10^{-9}\) & \(7.33379\cdot 10^{-13}\) \\
    500 & 1 & \(5.09336\cdot 10^{-8}\) & \(7.87879\cdot 10^{-12}\) \\
    500 & 2 & \(2.70421\cdot 10^{-7}\) & \(4.18307\cdot 10^{-11}\) \\
    500 & 3 & \(4.62975\cdot 10^{-6}\) & \(7.16164\cdot 10^{-10}\) \\
    500 & 4 & \(3.53666\cdot 10^{-5}\) & \(5.47076\cdot 10^{-9}\) \\
    500 & 5 & \(0.000144481\) & \(2.23495\cdot 10^{-8}\) \\
    500 & 6 & \(0.000741575\) & \(1.14712\cdot 10^{-7}\) \\
    500 & 7 & \(0.0285386\) & \(4.41456\cdot 10^{-6}\) \\
    500 & 8 & \(0.414411\) & \(6.41041\cdot 10^{-5}\) \\
    500 & 9 & \(5.06399\) & \(0.000783335\) \\
    \hline
    1000 & 0 & \(5.03021\cdot 10^{-8}\) & \(2.75309\cdot 10^{-12}\) \\
    1000 & 1 & \(2.34771\cdot 10^{-8}\) & \(1.28493\cdot 10^{-12}\) \\
    1000 & 2 & \(1.75109\cdot 10^{-6}\) & \(9.58395\cdot 10^{-11}\) \\
    1000 & 3 & \(4.60324\cdot 10^{-5}\) & \(2.51941\cdot 10^{-9}\) \\
    1000 & 4 & \(0.000152474\) & \(8.34511\cdot 10^{-9}\) \\
    1000 & 5 & \(0.000391032\) & \(2.14016\cdot 10^{-8}\) \\
    1000 & 6 & \(0.116013\) & \(6.34951\cdot 10^{-6}\) \\
    1000 & 7 & \(0.216661\) & \(1.18581\cdot 10^{-5}\) \\
    1000 & 8 & \(2.49188\) & \(0.000136384\) \\
    1000 & 9 & \(33.2835\) & \(0.00182164\)
  \end{longtable}
\end{center}
\subsubsection{Тестирование на матрицах Гильберта}
\begin{center}
  \begin{longtable}{l|l|l}
    \(n\) & \(\Vert x^* - x_k \Vert\) & \(\frac{\Vert x^* - x_k \Vert}{\Vert x^* \Vert}\) \\
    \hline
    10 & \(4.53743\cdot 10^{-8}\) & \(2.31249\cdot 10^{-9}\) \\
    50 & \(6.4107\cdot 10^{-7} \) & \(3.09422\cdot 10^{-9}\) \\
    100 & \(2.98603\cdot 10^{-6}\) & \(5.13347\cdot 10^{-9}\) \\
    500 & \(0.000384986\) & \(5.95524\cdot 10^{-8}\) \\
    1000 & \(0.00281381 \) & \(1.54003\cdot 10^{-7}\)
  \end{longtable}
\end{center}
\subsection{Метод сопряженных градиентов}
\subsubsection{Тестирование на матрицах с диагональным преобладание}
\begin{center}
  \begin{longtable}{l|l|l|l}
    \(n\) & \(\Vert x^* - x \Vert\) & \(\frac{\Vert x^* - x \Vert}{\Vert x^* \Vert}\) \\
    \hline
    10 & \(1.25957\cdot 10^{-12}\) & \(6.41935\cdot 10^{-14}\) \\
    50 & \(2.75509\cdot 10^{-6}\) & \(1.32978\cdot 10^{-8}\) \\
    100 & \(140.714\) & \(0.241911\) \\
    500 & \(2.93751\cdot 10^{-5}\) & \(4.54395\cdot 10^{-9}\) \\
    1000 & \(7.42658\cdot 10^{-5}\) & \(4.06466\cdot 10^{-9}\) \\
    10000 & \(0.00290357\) & \(5.02875\cdot 10^{-9}\) \\
  \end{longtable}
\end{center}
\subsubsection{Тестирование на матрицах с диагональным преобладание с обратным знаком недиагональных элементов}
\begin{center}
  \begin{longtable}{l|l|l|l}
    \(n\) & \(\Vert x^* - x \Vert\) & \(\frac{\Vert x^* - x \Vert}{\Vert x^* \Vert}\) \\
    \hline
    10 & \(1.52382\cdot 10^{-13}\) & \(7.76613\cdot 10^{-15}\) \\
    50 & \(1.73207\cdot 10^{-6}\) & \(8.36006\cdot 10^{-9}\) \\
    100 & \(6.43925\cdot 10^{-6}\) & \(1.10701\cdot 10^{-8}\) \\
    500 & \(4.07312\cdot 10^{-5}\) & \(6.3006\cdot 10^{-9}\) \\
    1000 & \(7.31752\cdot 10^{-5}\) & \(4.00497\cdot 10^{-9}\) \\
    10000 & \(0.00237227\) & \(4.10858\cdot 10^{-9}\) \\
    100000 & \(0.0372809\) & \(2.04194\cdot 10^{-9}\)
  \end{longtable}
\end{center}
\subsubsection{Тестирование на матрицах Гильберта}
\begin{center}
  \begin{longtable}{l|l|l|l}
    \(n\) & \(\Vert x^* - x \Vert\) & \(\frac{\Vert x^* - x \Vert}{\Vert x^* \Vert}\) \\
    \hline
    10 & 0.311773 & 0.0158894 \\
    50 & 1.48018 & 0.00714428 \\
    100 & 4.02852 & 0.00692568 \\
    500 & 53.2963 & 0.00824426 \\
    1000 & 144.311 & 0.0078983
  \end{longtable}
\end{center}
\section{Выводы}
\end{document}
