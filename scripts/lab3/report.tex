% Created 2021-04-20 Tue 19:43
% Intended LaTeX compiler: pdflatex

\documentclass[english]{article}
\usepackage[T1, T2A]{fontenc}
\usepackage[lutf8]{luainputenc}
\usepackage[english, russian]{babel}
\usepackage{minted}
\usepackage{graphicx}
\usepackage{longtable}
\usepackage{hyperref}
\usepackage{xcolor}
\usepackage{natbib}
\usepackage{amssymb}
\usepackage{stmaryrd}
\usepackage{amsmath}
\usepackage{caption}
\usepackage{mathtools}
\usepackage{amsthm}
\usepackage{tikz}
\usepackage{grffile}
\usepackage{extarrows}
\usepackage{wrapfig}
\usepackage{algorithm}
\usepackage{algorithmic}
\usepackage{lipsum}
\usepackage{rotating}
\usepackage{placeins}
\usepackage[normalem]{ulem}
\usepackage{amsmath}
\usepackage{textcomp}
\usepackage{capt-of}


\usepackage{geometry}
\geometry{a4paper,left=2.5cm,top=2cm,right=2.5cm,bottom=2cm,marginparsep=7pt, marginparwidth=.6in}
 \usepackage{hyperref}
 \hypersetup{
     colorlinks=true,
     linkcolor=blue,
     filecolor=orange,
     citecolor=black,      
     urlcolor=blue,
     }

\date{}
\title{}
\hypersetup{
 pdfauthor={},
 pdftitle={},
 pdfkeywords={},
 pdfsubject={},
 pdfcreator={Emacs 28.0.50 (Org mode 9.4.4)}, 
 pdflang={English}}
\begin{document}

\begin{titlepage}
    \begin{center}
        \large\textbf{Федеральное государственное автономное образовательное учреждение высшего образования ``Национальный исследовательский университет ИТМО``} \\
        \vspace{0.5cm}
        Факультет информационных технологий и программирования \\
        \vspace{0.5cm}
        Направление ``Прикладная математика и информатика`` \\
        \vspace{3cm}
        Отчет к лабораторной работе №3 \\
        \vspace{0.5cm}
        \textbf{Методы решения систем линейных уравнений}
    \end{center}
    \vfill
    \begin{flushright}
        \large
        Выполнили студенты группы М3237 \\
        \vspace{0.5cm}
        Ярошевский Илья \\
        Аникина Вероника \\
    Крюков Александр
    \end{flushright}
    \vspace{3cm}
    \begin{center}
        Санкт-Петербург 2021
    \end{center}
\end{titlepage}

\section{Цели работы}

\begin{enumerate}
    \item Реализовать прямой метод решения СЛАУ на основе \(LU\)-разложения
    \item Провести исследование метода на матрицах, число обусловленности которых регулируется за счёт изменения диагонального  преобладания 
    \item Провести исследование метода на матрицых Гильберта различной размерности
    \item Реализовать метод Гаусса с выбором ведущего элемента для плотных матриц
\end{enumerate}
\section{Ход работы}
\section{Выводы}
\end{document}
